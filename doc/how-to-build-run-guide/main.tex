\documentclass[10pt]{article}
\usepackage[utf8]{inputenc}
\usepackage{geometry}
 \geometry{
 a4paper,
 total={170mm,257mm},
 left=20mm,
 top=20mm,
 }

\title{BUILD-RUN-DOCUMENT}
\author{Saurabh Sharma}
\date{30 November 2018}

\usepackage{natbib}
\usepackage{graphicx}

\begin{document}

\maketitle
\noindent This document shows the steps needed to clone, build and run the FWI code and to install the prerequisite packages. 
\section{Pre-requisites}
The prerequisite development tools needed can be installed using the following commands.

\begin{enumerate}
    \item \texttt{sudo apt-get install git}  
    \item \texttt{sudo apt-get install qt5-default}
    \item \texttt{sudo apt-get install libeigen3-dev}
    \item \texttt{sudo apt-get install python2.7-dev}
    \item \texttt{sudo apt-get install python2.7}
    \item \texttt{sudo apt-get install python-tk}
    \item \texttt{sudo apt-get install python-numpy}
    \item \texttt{sudo apt-get install python-matplotlib}

\end{enumerate}

\section{Cloning the Repository}
\noindent To clone the FWI repository using git,
\newline
\texttt{git clone -o redmine https://git.alten.nl/parallelized-fwi.git}
\newline
This will create a copy of the repository, in a folder named \textbf{parallelized-fwi}
\newline
Any branch as needed can then be checked out from inside the \texttt{parallelized-fwi} folder, e.g. the develop branch
\newline
\texttt{git checkout develop}

\section{Build/Run}
To build the project, first create a folder titled \textbf{build} outside the \textbf{parallelized-fwi} folder.
\newline
\texttt{NOTE: This folder should be exactly 1 level outside the parallelized-fwi folder.}
\newline
\texttt{mkdir Build}
\newline
\texttt{cd Build}
\newline
\texttt{cmake -DCMAKE\_BUILD\_TYPE=Release ../parallelized-fwi/}
\newline
\texttt{make -j4} (the flag -j is used to build in parallel)
\newline
Now, the individual scripts for the preProcessing and the processing part can be run as shown below:
\newline
\texttt{cd applications}
\newline
\texttt{cd preProcessing}
\newline
\texttt{./FWIPreProcess}
\newline
\texttt{cd ../processing}
\newline
\texttt{./FWIProcess}
\newline
The input parameters for the code are provided in the input card i.e. default.in. User can create his/her own input card with a new name e.g. \texttt{newCard.in}. To use this input card use the card name as an argument when running the executables, \texttt{./FWIPreProcess newCard} and \texttt{./FWIProcess newCard}.

\newpage
\noindent For post-processing (i.e. generation of image using the estimated chi values), the python script \texttt{imageCreator\_CMake.py} can be used. This script is located inside the \textbf{parallelized-fwi} folder and can used as,
\newline
\texttt{python imageCreator\_CMake.py}
\newline
The pre-processing, processing and the image creation step can all be grouped together using the python wrapper \texttt{wrap\_FWI\_CMake.py} located inside the \textbf{parallelized-fwi} folder.
\newline
\texttt{python wrap\_FWI\_CMake.py}


\end{document}
