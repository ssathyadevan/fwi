\documentclass[10pt]{article}
%\usepackage[utf8]{inputenc} % Under score not copyable
\usepackage[T1]{fontenc}
\usepackage{geometry}
 \geometry{
 a4paper,
 total={170mm,257mm},
 left=20mm,
 top=20mm,
 }
 
\setlength\parindent{0pt}

\title{BUILD-RUN-DOCUMENT}
\author{Saurabh Sharma, Hugo Solera Licona, I\~{n}aki Martin Soroa, Noud van Herpen, \\
Chris Zevenbergen, Tianyuan Wang}
\date{\today}

\usepackage{natbib}
\usepackage{graphicx}
\usepackage{hyperref}
\usepackage{xcolor}

\begin{document}

\maketitle
\noindent This document shows the steps needed to install the prerequisite packages, clone, build, run, and do post-processing analysis for the FWI code. 

\section{Pre-requisites}
The prerequisite development tools needed can be installed using the following commands.\\

\noindent \texttt{sudo apt install git qt5-default libeigen3-dev eog cmake python3.7}\\
\texttt{sudo apt install python3.7-dev python3-tk python3-numpy python3-matplotlib python3-skimage}  


\section{Cloning the Repository}
\noindent To clone the FWI repository using git, 
\newline

\texttt{git clone -o redmine https://git.alten.nl/parallelized-fwi.git} 
\newline

This will create a copy of the repository, in a folder named \textit{parallelized-fwi}. Any branch as needed can then be checked out from inside the \textit{parallelized-fwi} folder, e.g. the develop branch: 
\newline

\texttt{git checkout develop}

\section{Build}
In this section, the process to build is explained. It is worth noticing that there is a \texttt{Build} and a \texttt{BuildAndRun} python scripts which does the build of the project automatically. 


\subsection{Install Google Test}
First we need to install Google Test using the following commands:
\newline

\texttt{sudo apt-get install libgtest-dev}
\newline
\texttt{cd /usr/src/gtest}
\newline
\texttt{sudo cmake CMakeLists.txt}
\newline
\texttt{sudo make}
\newline
\texttt{sudo cp *.a /usr/lib}

%First go back to your home directory, then make the googletest directory. Then download Google test from Github. Then execute  cmake, make and make install commands. Finally set the gtest root to the path of the working directory \texttt{\$/googletest/install}.
%\newline
%\texttt{cd ..}
%\newline
%\texttt{mkdir googletest}
%\newline
%\texttt{cd googletest}
%\newline
%\texttt{git clone https://github.com/google/googletest source}
%\newline
%\texttt{mkdir build}
%\newline
%\texttt{cd build}
%\newline
%\texttt{cmake = -DCMAKE\_BUILD\_TYPE=Release -DCMAKE\_INSTALL\_PREFIX=\textasciitilde/googletest/install/ ../source}
%\newline
%\texttt{make}
%\newline
%\texttt{make install}
%\newline
%\texttt{cd ..}
%\newline
%\texttt{cd install/}
%\newline
%\texttt{export GTEST\_ROOT=\$PWD}
%\newline
%In order to set the \texttt{GTEST\_ROOT} reference more permanent one, one has to add the \texttt{GTEST\_ROOT} to bashrc file by using vim or another editior.
%\newline
%\texttt{cd}
%\newline
%\texttt{vim \textasciitilde/.bashrc}
%\newline
%Then, press i or insert to modify the file and add the following line to the file: 
%\newline
%\texttt{export GTEST\_ROOT=/googletest/install}
%\newline
%Then, exit insert  mode by pressing esc. To save bashrc and exit vim, enter the \texttt{:wq} command. Then, in order to take the changes in bashrc into effect enter:
%\newline
%\texttt{source \textasciitilde/.bashrc}

\subsection{Build}
To build the project, first create a folder titled \textit{build} outside the \textit{parallelized-fwi} folder. Afterwards the code is built and run. 
NOTE: This folder should be exactly 1 level outside the parallelized-fwi folder.
\newline

\texttt{cd \textasciitilde}
\newline
\texttt{mkdir Build}
\newline
\texttt{cd Build}
\newline
\texttt{cmake -DCMAKE\_BUILD\_TYPE=Release -DCMAKE\_INSTALL\_PREFIX=../FWIInstall ../parallelized-fwi/}
\newline
\texttt{make install} 
\newline

\noindent The first flag in the cmake command above enforces that the release version of the code be built and the 2nd flag implies that all the executables of the program will be placed in a folder \textit{FWIInstall} (executables are in \textit{FWIInstall/bin})

\section{Run}
To run, first check out the \textit{inputFiles} folder in the parallelized-fwi folder and copy the \textit{default} folder to the \textit{FWIInstall} folder. This can be done by issuing the following command: 
\newline

\texttt{cp -r ../parallelized-fwi/inputFiles/default/ ../FWIInstall} 
\newline

Also, make sure that there is a \textit{bin} folder in \textit{FWIInstall}, if there is not you can copy it from \textit{Build/runtime/}. You can do this with the command: 
\newline

\texttt{cp -r ../Build/runtime/bin/ ../FWIInstall} 
\newline

Now, the individual scripts for the preProcessing and the processing part can be run as shown below: \newline

\texttt{cd \textasciitilde/FWIInstall/bin/} \newline
\texttt{./FWI\_PreProcess ../default/} \newline
\texttt{./FWI\_UnifiedProcess ../default/ conjugateGradientInversion integralForwardModel}
\newline

Choose either integralForwardModel or finiteDifferenceForwardModel.
\newline

\noindent For both executables we have to pass one argument which is the path to the case to be run (in this case \textit{../default/}. It is important to notice that for the UnifiedProcess, a second argument is needed, this argument corresponds to the name of the inversion method you want to use. Note that the arguments are the locations relative to the executable position. The default input parameters are stored in the \textit{default/} which is located in \textit{\textasciitilde/parallelized-fwi/inputFiles/}. Users can create their own set of input cards by copying the default folder into a new folder, modifying the input cards and giving the new folder's location argument to both PreProcess and Process.\\

%Note: It is very useful to add the \textit{$\sim$/FWIInstall/bin/} folder to the PATH environment variable. This means that both the \texttt{FWI\_PreProcess} and \texttt{FW\_Process} commands can be run from any folder without having to give the path to where they can be found. This can be achieved by issuing the following commands. \\
%\texttt{vim $\sim$/.bashrc}\\
%Add the following line at the end of this file (don't forget the colon!).\\
%\texttt{export PATH=\$PATH:$\sim$/FWIInstall/bin/}\\
%Now execute the following command in your terminal.\\
%\texttt{source $\sim$/.bashrc}\\
%Now you can run \texttt{FWI\_PreProcess} and \texttt{FW\_Process} commands from anywhere in your terminal!

\section{Post-Processing}
For post-processing (i.e. generation of image using the estimated chi values and the residual plot), the python script \texttt{postProcessing-python3.py} can be used. Note that when using the \texttt{BuildAndRun} and \texttt{Run} scripts, the postprocessing is done automatically. The \texttt{postProcessing-python3.py} script is located inside the \textit{parallelized-fwi/pythonScripts} folder. This can be copied to the FWIInstall folder and then executed using 
the following commands:
\newline

\texttt{cp \textasciitilde/parallelized-fwi/pythonScripts/postProcessing-python3.py ..} \newline
\texttt{cd ..} \newline
\texttt{python3 postProcessing-python3.py default/} 
\newline

The run case folder is provided as the argument for the python script. The pre-processing, processing and the post-processing can all be grouped together using the python wrapper \texttt{wrapper.py} located inside the \textit{parallelized-fwi/pythonScripts} folder.
\newline

\noindent The postprocessed data can then be visualized using EOG from the output folder: \\ \newline
\texttt{cd default/output/} \newline
\texttt{eog defaultResult.png}
\newline
 
\section{Unit Test}

Unit tests are written in one file per class. The file must be located in the /texttt{test} directory of the respective library. To make the unit tests available for Google Test, make sure \texttt{CMakeLists.txt} in \texttt{test} has the following line:
\newline

\texttt{build\char`_test(TARGET myClassTest SOURCE myClassTest.cpp LIBRARIES core\char`_library)}
\newline

In this case, myClassTest covers a class that is part of the core library.\\
After building, the test can be ran by writing in the terminal:\\
\newline

\texttt{cd ../../bin/test}
\newline
\texttt{./myClassTest}
\newline

\section{Regression Test}
This section details the comparison of a \textit{default} run with the \textit{fast} regression data. In order to run a regression test the following steps need to be taken. First, a \textit{test} folder is created, then the contents of the regression data folder, the default folder, the input cards and the regression test python scripts are copied to this \textit{test} folder. Then finally the python scripts can be run in succession.  
\newline

\texttt{cd \textasciitilde/FWIInstall}
\newline
\texttt{mkdir test}
\\
\texttt{cp -r \textasciitilde/parallelized-fwi/tests/regression\_data/fast/ test/}
\newline
\texttt{cp \textasciitilde/parallelized-fwi/tests/testScripts/*.py test/}
\newline
\texttt{cp -r default/ test/}
\\
\texttt{cd test/}
\\
\texttt{python3 regressionTestPreProcessing\_python3.py fast default}
\newline
\texttt{python3 regressionTestProcessing\_python3.py fast default}

\section{Troubleshooting}
\subsection{SSH Tunneling}
In case the Ubuntu Virtual Machine is very slow and has long response times, try setting up an SSH connection between the \textbf{Ubuntu host} \& \textbf{Windows guest}. This way, you can use Visual Studio Code for development with fast response times from Windows. \href{https://stackoverflow.com/questions/5906441/how-to-ssh-to-a-virtualbox-guest-externally-through-a-host}{\color{blue}{This post}} explains how to set up such a connection.
\begin{enumerate}
    \item In the \textbf{host} Ubuntu virtual machine toolbar (make sure full-screen view is disabled), go to \texttt{Devices > Network > Network Settings > Advanced > Port Forwarding}
    \item Add a new port forwarding rule with the following parameters: \\
    \begin{tabular}{|l|l|l|l|l|l|}
        \hline
        Name & Protocol & Host IP & Host Port & Guest IP & Guest Port \\ \hline
        ssh & TCP & & 3022 & & 22 \\ \hline 
    \end{tabular}
    \item In the \textbf{host}, install an ssh server: \\
    \texttt{sudo apt-get install openssh-server}
    \item In the \textbf{guest}, install Visual Studio Code with the Remote Development extension.
    \item In the \textbf{guest}, open cmd and write: \\
    \texttt{ssh -p 3022 user@127.0.0.1} \\
    Where \texttt{user} is the account in the virtual machine, before the '@' sign, in the terminal.
    \item Run Visual Studio Code \& click on the Remote Development icon in the bottom-left corner.
    \item From the drop-down menu, go to Remote-SSH: Open Configuration File. Select one, remember which one.
    \item Add to this document the following lines: \\
    \texttt{Host 127.0.0.1} \\
    \texttt{HostName 127.0.0.1} \\
    \texttt{Port 3022} \\
    \texttt{User user} \\
    Where \texttt{user} is the account in the virtual machine, before the '@' sign.
    \item Click the Remote Development button in the bottom-left corner again \& select Remote-SSH: Connect To Host. Select the SSH configuration file you just modified.
    \item Don't forget to provide the \textbf{host} password in an easily overlooked password field, located at the top of the editor.
\end{enumerate}

\end{document}

