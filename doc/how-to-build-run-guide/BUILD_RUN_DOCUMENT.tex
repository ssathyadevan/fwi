\documentclass[10pt]{article}
\usepackage[utf8]{inputenc}
\usepackage{geometry}
 \geometry{
 a4paper,
 total={170mm,257mm},
 left=20mm,
 top=20mm,
 }

\title{BUILD-RUN-DOCUMENT}
\author{Saurabh Sharma}
\date{30 November 2018}

\usepackage{natbib}
\usepackage{graphicx}

\begin{document}

\maketitle
\noindent This document shows the steps needed to clone, build and run the FWI code and to install the prerequisite packages. 
\section{Pre-requisites}
The prerequisite development tools needed can be installed using the following commands.

\begin{enumerate}
    \item \texttt{sudo apt-get install git}  
    \item \texttt{sudo apt-get install qt5-default}
    \item \texttt{sudo apt-get install libeigen3-dev}
    \item \texttt{sudo apt-get install python3.7-dev}
    \item \texttt{sudo apt-get install python3.7}
    \item \texttt{sudo apt-get install python3-tk}
    \item \texttt{sudo apt-get install python3-numpy}
    \item \texttt{sudo apt-get install python3-matplotlib}
  	\item \texttt{sudo apt-get install eog}
  	\item \texttt{sudo apt-get install cmake}

\end{enumerate}


\section{Cloning the Repository}
\noindent To clone the FWI repository using git,
\newline
\texttt{git clone -o redmine https://git.alten.nl/parallelized-fwi.git}
\newline
This will create a copy of the repository, in a folder named \textit{parallelized-fwi}
\newline
Any branch as needed can then be checked out from inside the \textit{parallelized-fwi} folder, e.g. the develop branch
\newline
\texttt{git checkout develop}

\section{Build/Run}
In this section the process to build the code is explained in two steps, namely downloading and installing Google Test and thereafter building the code and finally running it.  


\subsection{Install Google Test}
First go back to your home directory, then make the googletest directory. Then download Google test from Github. Then execute  cmake, make and make install commands. Finally set the gtest root to the path of the working directory \texttt{\$/googletest/install}.
\newline
\texttt{cd ..}
\newline
\texttt{mkdir googletest}
\newline
\texttt{cd googletest}
\newline
\texttt{git clone https://github.com/google/googletest source}
\newline
\texttt{mkdir build}
\newline
\texttt{cd build}
\newline
\texttt{cmake = -DCMAKE\_BUILD\_TYPE=Release -DCMAKE\_INSTALL\_PREFIX=\textasciitilde/googletest/install/ ../source}
\newline
\texttt{make}
\newline
\texttt{make install}
\newline
\texttt{cd ..}
\newline
\texttt{cd install/}
\newline
\texttt{export GTEST\_ROOT=\$PWD}
\newline
In order to set the \texttt{GTEST\_ROOT} reference more permanent one, one has to add the \texttt{GTEST\_ROOT} to bashrc file by using vim or another editior.
\newline
\texttt{cd}
\newline
\texttt{vim \textasciitilde/.bashrc}
\newline
Then, press i or insert to modify the file and add the following line to the file: 
\newline
\texttt{export GTEST\_ROOT=/googletest/install}
\newline
Then, exit insert  mode by pressing esc. To save bashrc and exit vim, enter the \texttt{:wq} command. Then, in order to take the changes in bashrc into effect enter:
\newline
\texttt{source \textasciitilde/.bashrc}
\newline
\noindent Another way of installing GTest
\newline
\texttt{sudo apt-get install libgtest-dev}
\newline
\texttt{cd /usr/src/gtest}
\newline
\texttt{sudo cmake CMakeLists.txt}
\newline
\texttt{sudo make}
\newline
\texttt{sudo cp *.a /usr/lib}
\newline
\subsection{Build}
To build the project, first create a folder titled \textit{build} outside the \textit{parallelized-fwi} folder. Afterwards the code is built and run. 
\newline
\texttt{NOTE: This folder should be exactly 1 level outside the parallelized-fwi folder.}
\newline
\texttt{mkdir Build}
\newline
\texttt{cd Build}
\newline
\texttt{cmake -DCMAKE\_BUILD\_TYPE=Release -DCMAKE\_INSTALL\_PREFIX= \textasciitilde/FWIInstall ../parallelized-fwi/}
\newline
\texttt{make install} 
\newline

\noindent The first flag in the cmake command above enforces that the release version of the code be built and the 2nd flag implies that all the executables of the program will be placed in a folder \textit{FWIInstall} (executables are in \textit{FWIInstall/bin})

\subsection{Run}
First enter the \textit{inputFiles} folder parallelized-fwi folder and copy its contents to a folder named \textit{input} inside \textit{FWIInstall} folder. Also, a folder named \textit{output} is created to store all the output files of the program.
\newline
\texttt{cd FWIInstall}
\newline
\texttt{mkdir input output}
\newline
\texttt{cp ../parallelized-fwi/inputFiles/* input/} 
\newline
Now, the individual scripts for the preProcessing and the processing part can be run as shown below:
\newline
\texttt{cd FWIInstall/bin}
\newline
\texttt{./FWI\_PreProcess ../input/ ../output/ default}
\newline
\texttt{./FWI\_Process ../input/ ../output/ default}
\newline
\noindent In the both the above executables the 1st argument (i.e. ../input) provides the directory where the input card and the temple/skull chi values are located , the 2nd argument (i.e. ../output) provides the directory where all the output files from each application will be stored and the 3rd argument is the name of the input card (which is default in this case). Also note that the first 2 arguments are the locations relative to the executable position. The default input parameters for the code are provided in the input card i.e. default.in. User can create his/her own input card with a new name e.g. \texttt{newCard.in}.

\newpage
\noindent For post-processing (i.e. generation of image using the estimated chi values and the residual plot), the python script \texttt{postProcessing.py} can be used. This script is located inside the \textit{parallelized-fwi/pythonScripts} folder. This can be copied to the FWIInstall folder and then used as,
\newline
\texttt{cp parallelized-fwi/pythonScipts/postProcessing.py FWIInstall/}
\newline
\texttt{cd FWIInstall}
\newline
\texttt{python3 postProcessing.py output/}
\newline
The folder where all the output files are located is provided as the argument for the python script. The pre-processing, processing and the post-processing can all be grouped together using the python wrapper \texttt{wrapper.py} located inside the \textit{parallelized-fwi/pythonScripts} folder.

\noindent The postprocessed data can then be visualized using EOG from the output folder:
\newline
\texttt{cd output}
\newline
\texttt{eog defaultResult.png}
\newline
 
\section{Unit Test}

The unit test executable is stored in the \textit{FWIInstall/bin} folder after the make install command. The following command can be used to run the test.
\newline
\texttt{cd FWIInstall/bin}
\newline
\texttt{./unittest}


\section{Regression Test}

This section details the comparison of a \textit{default} run with the \textit{fast} regression data. In order to run a regression test the following steps need to be taken. First, a \textit{test} folder is created, then the contents of a regression data folder, the output folder, the input card (default.in) and the regression test python scripts are copied to this \textit{test} folder. Then finally the python scripts can be run in succession.  
\newline
\texttt{mkdir test}
\newline
\texttt{cp parallelized-fwi/tests/regression\_data/fast/* test/}
\newline
\texttt{cp parallelized-fwi/tests/pythonScripts/* test/}
\newline
\texttt{cp input/default.in test/}
\newline
\texttt{cp output/* test/}
\newline
\texttt{cd test}
\newline
\texttt{python3 regressionTestPreProcessing.py fast default}
\newline
\texttt{python3 regressionTestProcessing.py fast default}

\end{document}

