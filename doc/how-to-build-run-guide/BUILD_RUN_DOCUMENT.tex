\documentclass[10pt]{article}
\usepackage[utf8]{inputenc}
\usepackage{geometry}
 \geometry{
 a4paper,
 total={170mm,257mm},
 left=20mm,
 top=20mm,
 }
\title{BUILD-RUN-DOCUMENT}
\author{Saurabh Sharma, I\~naki Mart\'in Soroa}
 \date{Last update: 17 October 2019}
\usepackage{natbib}
\usepackage{graphicx}
\begin{document}
\maketitle
\noindent This document shows the steps needed to clone, build and run the FWI code and to install the prerequisite packages. 
\section{Pre-requisites}
The prerequisite development tools needed can be installed using the following commands.\\
\noindent \texttt{sudo apt install git qt5-default libeigen3-dev python3.7 python3.7-dev python3-tk}\\
\texttt{sudo apt install python3-numpy python3-matplotlib eog cmake}  
%    \item \texttt{sudo apt-get install qt5-default}
%    \item \texttt{sudo apt-get install libeigen3-dev}
%    \item \texttt{sudo apt-get install python3.7-dev}
%    \item \texttt{sudo apt-get install python3.7}
%    \item \texttt{sudo apt-get install python3-tk}
%    \item \texttt{sudo apt-get install python3-numpy}
%    \item \texttt{sudo apt-get install python3-matplotlib}
%  	\item \texttt{sudo apt-get install eog}
%  	\item \texttt{sudo apt-get install cmake}
\section{Cloning the Repository}
\noindent To clone the FWI repository using git,
\newline
\texttt{git clone -o redmine https://git.alten.nl/parallelized-fwi.git}
\newline
This will create a copy of the repository, in a folder named \textit{parallelized-fwi}
\newline
Any branch as needed can then be checked out from inside the \textit{parallelized-fwi} folder. This document has been verified for the gradientDescent branch, later branches have added OpenCL dependencies, making this guide obsolete. In this example we use the gradientDescent branch.
\newline
\texttt{git checkout gradientDescent}
\section{Build/Run}
In this section the process to build the code is explained in two steps, namely downloading and installing Google Test and thereafter building the code and finally running it.  
\subsection{Install Google Test}
First we need to install Google Test using the following commands:\\
\texttt{sudo apt-get install libgtest-dev}
\newline
\texttt{cd /usr/src/gtest}
\newline
\texttt{sudo cmake CMakeLists.txt}
\newline
\texttt{sudo make}
\newline
\texttt{sudo cp *.a /usr/lib}
%First go back to your home directory, then make the googletest directory. Then download Google test from Github. Then execute  cmake, make and make install commands. Finally set the gtest root to the path of the working directory \texttt{\$/googletest/install}.
%\newline
%\texttt{cd ..}
%\newline
%\texttt{mkdir googletest}
%\newline
%\texttt{cd googletest}
%\newline
%\texttt{git clone https://github.com/google/googletest source}
%\newline
%\texttt{mkdir build}
%\newline
%\texttt{cd build}
%\newline
%\texttt{cmake = -DCMAKE\_BUILD\_TYPE=Release -DCMAKE\_INSTALL\_PREFIX=\textasciitilde/googletest/install/ ../source}
%\newline
%\texttt{make}
%\newline
%\texttt{make install}
%\newline
%\texttt{cd ..}
%\newline
%\texttt{cd install/}
%\newline
%\texttt{export GTEST\_ROOT=\$PWD}
%\newline
%In order to set the \texttt{GTEST\_ROOT} reference more permanent one, one has to add the \texttt{GTEST\_ROOT} to bashrc file by using vim or another editior.
%\newline
%\texttt{cd}
%\newline
%\texttt{vim \textasciitilde/.bashrc}
%\newline
%Then, press i or insert to modify the file and add the following line to the file: 
%\newline
%\texttt{export GTEST\_ROOT=/googletest/install}
%\newline
%Then, exit insert  mode by pressing esc. To save bashrc and exit vim, enter the \texttt{:wq} command. Then, in order to take the changes in bashrc into effect enter:
%\newline
%\texttt{source \textasciitilde/.bashrc}
\subsection{Build}
To build the project, first create a folder titled \textit{build} outside the \textit{parallelized-fwi} folder. Afterwards the code is built and run. NOTE: This folder should be exactly 1 level outside the parallelized-fwi folder.
\newline
\texttt{cd \textasciitilde}
\newline
\texttt{mkdir Build}
\newline
\texttt{cd Build}
\newline
\texttt{sudo cmake -DCMAKE\_BUILD\_TYPE=Release -DCMAKE\_INSTALL\_PREFIX= \textasciitilde/FWIInstall ../parallelized-fwi/}
\newline
\texttt{sudo make install} 
\newline
\noindent The first flag in the cmake command above enforces that the release version of the code be built and the 2nd flag implies that all the executables of the program will be placed in a folder \textit{FWIInstall} (executables are in \textit{FWIInstall/bin})
\subsection{Run}
First check out the \textit{inputFiles} folder in the parallelized-fwi folder and copy the \textit{default} folder to the \textit{FWIInstall} folder. This can be done by issuing the following command:
\newline
\texttt{cp -r ../parallelized-fwi/inputFiles/default/ \textasciitilde/FWIInstall} 
\newline
\texttt{cp -r ../Build/runtime/bin/ \textasciitilde/FWIInstall/bin} 
\newline
Now, the individual scripts for the preProcessing and the processing part can be run as shown below:
\newline
\texttt{cd \textasciitilde/FWIInstall/bin/}
\newline
\texttt{./FWI\_PreProcess ../default/}
\newline
\texttt{./FWI\_Process ../default/}
\newline
\noindent For both executables we have to pass one argument which is the path to the case to be run (in this case \textit{../default/}. Note that the arguments are the locations relative to the executable position. The default input parameters are stored in the \textit{default/} which is located in \textit{\textasciitilde/parallelized-fwi/inputFiles/}. Users can create their own set of input cards by copying the default folder into a new folder, modifying the input cards and giving the new folder's location argument to both PreProcess and Process.\\
%Note: It is very useful to add the \textit{$\sim$/FWIInstall/bin/} folder to the PATH environment variable. This means that both the \texttt{FWI\_PreProcess} and \texttt{FW\_Process} commands can be run from any folder without having to give the path to where they can be found. This can be achieved by issuing the following commands. \\
%\texttt{vim $\sim$/.bashrc}\\
%Add the following line at the end of this file (don't forget the colon!).\\
%\texttt{export PATH=\$PATH:$\sim$/FWIInstall/bin/}\\
%Now execute the following command in your terminal.\\
%\texttt{source $\sim$/.bashrc}\\
%Now you can run \texttt{FWI\_PreProcess} and \texttt{FW\_Process} commands from anywhere in your terminal!
\noindent For post-processing (i.e. generation of image using the estimated chi values and the residual plot), the python script \texttt{postProcessing.py} can be used. This script is located inside the \textit{parallelized-fwi/pythonScripts} folder. This can be copied to the FWIInstall folder and then executed using the following commands:
\newline
\texttt{cd \textasciitilde/FWIInstall}
\newline
\texttt{cp \textasciitilde/parallelized-fwi/pythonScipts/postProcessing-python3.py .}
\newline
\texttt{python3 postProcessing-python3.py default/}
\newline
The run case folder is provided as the argument for the python script. The pre-processing, processing and the post-processing can all be grouped together using the python wrapper \texttt{wrapper.py} located inside the \textit{parallelized-fwi/pythonScripts} folder.
\noindent The postprocessed data can then be visualized using EOG from the output folder:
\newline
\texttt{cd default/output/}
\newline
\texttt{eog defaultResult.png}
\newline
 
\section{Unit Test}
The unit test executable is stored in the \textit{FWIInstall/bin} folder after the make install command. The following commands can be used to run the test.
\newline
\texttt{cd ../../bin}
\newline
\texttt{./unittest}
\section{Regression Test}
This section details the comparison of a \textit{default} run with the \textit{fast} regression data. In order to run a regression test the following steps need to be taken. First, a \textit{test} folder is created, then the contents of the regression data folder, the default folder, the input cards and the regression test python scripts are copied to this \textit{test} folder. Then finally the python scripts can be run in succession.  
\newline
\texttt{cd ..}
\newline
\texttt{mkdir test}
\\
\texttt{cp -r \textasciitilde/parallelized-fwi/tests/regression\_data/fast/ test/}
\newline
\texttt{cp \textasciitilde/parallelized-fwi/tests/testScripts/*.py test/}
\newline
\texttt{cp -r default/ test/}
\\
\texttt{cd test/}
\\
\texttt{python3 regressionTestPreProcessing\_python3.py fast default}
\newline
\texttt{python3 regressionTestProcessing\_python3.py fast default}
\end{document}
