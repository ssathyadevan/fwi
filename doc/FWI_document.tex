\documentclass[10pt,a4paper]{article}
\usepackage{geometry}
\usepackage{graphicx}
\usepackage{amsthm,amsfonts,mathrsfs}
\usepackage{enumerate}
\usepackage{amsmath,accents}
\usepackage{amsfonts}
\usepackage{amssymb}
\usepackage{relsize}
\usepackage{latexsym}
\usepackage{subfigure}
\usepackage{hyperref}
\usepackage{nomencl}


\setlength{\parindent}{0pt}

\newcommand{\nomunit}[1]{\renewcommand{\nomentryend}{\hspace*{\fill}#1}}

\newcommand{\partder}[2]{\ensuremath{\frac{\partial #1}{\partial #2}}}
\newcommand{\secpartder}[2]{\ensuremath{\frac{\partial^2 #1}{\partial #2^2}}}
\newcommand{\nthpartder}[3]{\ensuremath{\frac{\partial^{#1} #2}{\partial #3^{#1}}}}
\newcommand{\fullder}[2]{\ensuremath{\frac{\mbox{d} #1}{\mbox{d} #2}}}
\newcommand{\secfullder}[2]{\ensuremath{\frac{\mbox{d}^2 #1}{\mbox{d} #2^2}}}
\newcommand{\nfullder}[3]{\ensuremath{\frac{\mbox{d}^{#1} #2}{\mbox{d} #3^{#1}}}}
\newcommand{\mixedder}[3]{\ensuremath{\frac{\partial^{2} #1}{\partial #2 \partial #3}}}
\newcommand{\df}[1]{\, \ensuremath{\mbox{d}#1}}
\newcommand{\grad}{\, \mbox{grad} \,}
\newcommand{\dive}{\, \mbox{div} \,}
\newcommand{\real}[1]{\text{Re} \left\{ #1 \right\}}
\newcommand{\imag}[1]{\text{Im}\left\{ #1 \right\}}

\newcommand{\xs}{\mathbf{x}_\text{s}}
\newcommand{\xr}{\mathbf{x}_\text{r}}
\newcommand{\x}{\mathbf{x}}

\newtheorem{thm}{Theorem}[section]
\makenomenclature
%opening
\title{Full Waveform Inversion}
\author{ALTEN Netherlands}

\begin{document}

\maketitle
\section{Introduction}

The Full Wave inversion is a complex imaging technique which can be achieved by illuminating the interior of the object with e.g. acoustic or electromagnetic waves, while receivers placed around the object measure the response. From these measurements an image of the properties of the object's interior can be derived.

It has various applications including,
 
\begin{enumerate}
    \item seismology
    \item weld testing
    \item non-destructive testing - ground radar
    \item medical: ultrasonic applications
\end{enumerate}

This document explains an overview of the equations employed, the computational model and the code algorithm.

\section{Mathematical Model}

The mathematical model outlines the equations used in the code for the FWI. We begin with the simplest acoustic equation - the Helmholtz equation. 
For the acoustic case with constant density, the wave equation can be described by a set of two couples equations, being the data equation and the object equation. The data equation describes the seismic dataset, in terms of a total field $p_{\text{tot}}$ at each grid point int he subsurface, the contrast function $\chi$ and the Green's function $G$ in a background medium:

\begin{equation} \label{eq:dataequation}
p_\text{data}(\mathbf{x_\text{r}},\mathbf{x_\text{s}},\omega) = \int \mathcal{G}(\xr, \x, \omega) p_\text{tot}(\x, \xs, \omega) \chi(\x) \df{\x} \end{equation}

Reading from right to left, equation (\ref{eq:dataequation}) can be understood as follows. A source transmits a wavefield that propagates to every point in the subsurface. Note that this wavefield $p_\text{tot}$ is generally quite complex because it takes the interaction of all scatterers in the subsurface already into account. This wavefield is creating secondary sources in all points where the contrast function $\chi$ is non-zero. The secondary sources transmit energy through a smooth background medium to the receivers, represented by the Green's function G in equation (\ref{eq:dataequation}) . The measured seismic data at every receiver are then a summation of all secondary sources. It should be mentioned that direct waves, including ground roll and surface waves are supposed to be removed from the measured data to obtain $p_\text{data}$.
We use the 2-D case for the Helmholtz equation. Further, the Green's function is calculated for this equation. The contrast function has to be determined. Further the conjugate scheme is established to determine the error functional. 

Measured seismic data always contains some form of noise, so the inversion process is required to be regularised. Therefore, we extend the conjugate gradient scheme in a way that it contains a multiplicative regularisation factor. 

\subsection{Helmholtz equation}

The simplest model we can use is the acoustic Helmholtz equation. We use a scalar pressure field $u(\mathbf{x})$ and the domain is modeled using a scalar quantity $\chi(\mathbf{x})$. This quantity is called and contrast and can be directly related to the wave speed at that point in the domain. Mathematically the equation has the form

\begin{equation} \label{eq:helmholtz}
\nabla^2 u(\mathbf{x}) + k(\mathbf{x})^2 u(\mathbf{x}) = -f_{\text{ext}}(\mathbf{x}).
\end{equation} 

The wave number $k$ is given by $k = \frac{\omega}{c}$. We split the pressure field into $u(\mathbf{x}) = u_0(\mathbf{x}) + u_{\text{ind}}(\mathbf{x})$, where $u_0(\mathbf{x})$ is defined as the field given by the background velocity and external sources,

\begin{equation} \label{eq:reference}
\nabla^2 u_0(\mathbf{x}) + k_0^2 u_0(\mathbf{x}) = -f_{\text{ext}}(\mathbf{x}).
\end{equation} 

Substituting in (\ref{eq:helmholtz}) results in

\begin{equation} \label{eq:induced}
\nabla^2 u_\text{ind}(\mathbf{x}) + k_0^2 u_\text{ind}(\mathbf{x}) = -f_\text{ind}(\mathbf{x}),
\end{equation} 

with $f_\text{ind}(\mathbf{x}) = k_0^2 \chi(\mathbf{x}) u(\mathbf{x})$ the induced source term.

\subsection{The Green's function}

The Green's function of this equation is defined as the solution $G(\mathbf{x}, \mathbf{y})$, of the equation

\[ \nabla^2 G(\mathbf{x}, \mathbf{y}) + k_0^2 G(\mathbf{x}, \mathbf{y}) = -\delta(\mathbf{x} - \mathbf{y}). \]

The solution to (\ref{eq:induced}) is then given by 

\[ u_\text{ind}(\mathbf{x}) = \int_{\mathbf{y} \in \mathbb{R}^n} G(\mathbf{x}, \mathbf{y}) f_\text{ind}(\mathbf{y}) \df{\mathbf{y}}.\footnote{Notice that we assume that $f_\text{ind}$ has compact support so the integral is over a finite region.} \]

The interesting cases are the 2D and 3D case. We will first focus on the 2D case. The Green's function is then given by

\[ G(\mathbf{x}, \mathbf{y}) = \frac{\imath}{4} H_0^{(1)}(k_0 \|\mathbf{x} - \mathbf{y}\|) = -\frac{1}{4} Y_0(k_0 \|\mathbf{x} - \mathbf{y}\|) + \frac{\imath}{4} J_0(k_0 \|\mathbf{x} - \mathbf{y}\|).  \]

\subsection{The Contrast function}

The contrast function is defined as: 

\[ \chi(\mathbf{x}) = 1 - \left(\frac{c_0(\vec{x})}{c(\vec{x})} \right)^2. \]

It depends on the difference between a known background medium $c_\text{0}(\vec{x})$ and the true, but unknown, subsurface model $c(\vec{x})$. The total field on the right-hand side in equation  is dependent on the contrast (\ref{eq:dataequation}), because it contains the interaction between all subsurface scatterers. Then it follows that there is a nonlinear relationship between the subsurface properties and the measured seismic data.

Notice that the contrast is generally unknown, and will be approximated using an iterative scheme. During each step the induced source is considered constant and known so we basically solve the same equation as (\ref{eq:reference}) each step.

\subsection{The Conjugate Gradient Scheme}

In the inversion scheme, we try to minimise the difference between the measured data $p_\text{data}$ and the modelled data that is obtained using the currently best estimate of the contrast function and the fixed total field. 

The residual between the measured and modelled data is obtained as: 

\[ r(\xr, \xs, \omega) = p_{\text{data}}(\xr, \xs, \omega) - \left[\mathcal{K}_\chi \right](\xr, \xs, \omega), \]

with

\[ \left[\mathcal{K}_\chi \right](\xr, \xs, \omega) = \int \mathcal{G}(\xr, \x, \omega) p_\text{tot}(\x, \xs, \omega) \chi(\x) \df{\x} \]

a linear operator in $\chi$. 

The error functional is equal to

\[ F(\chi) = \eta \int |r(\xr, \xs, \omega)|^2 \df{\mathbf{x}_\text{s}} \df{\xr} \df{\omega}, \]

with

\[ \eta^{-1} = \int | p_\text{data}(\xr, \xs, \omega) |^2 \df{\xs} \df{\xr} \df{\omega}, \]

so that for $\chi = 0$ we have $F = 1$. Notice the implicit dependency on $\chi$.

We want to find a sequence of contrast functions, $\chi^{(n)}(\vec{x}), n = 1,2,...,$ in which error functional decreases with increasing iterations. Therefore, after each iteration the contrast function is updated as: 

\[ \chi^{(n)}(\vec{x}) =  \chi^{(n-1)}(\vec{x}) + \alpha_n\zeta_n(\vec{x}) \]

Here, the step size of the update is determined by the parameter $\alpha_n$ while the update directions are conjugate gradient directions given by

\[\zeta_1(\vec{x}) = g_1(\vec{x}) ,\zeta_n(\vec{x}) = g_n(\vec{x}) + \gamma_n\zeta_{n-1}(\vec{x}) \]

The functional derivative w.r.t. $\chi$ in direction $\mathbf{d}$ of the error functional is equal to

\begin{eqnarray*}
\partder{F(\mathbf{\chi}, \mathbf{d})}{\mathbf{\chi}} & = & \lim_{\epsilon \rightarrow 0} \frac{F(\mathbf{\chi} + \epsilon \mathbf{d}) - F(\mathbf{\chi})}{\epsilon} \\
& = & \lim_{\epsilon \rightarrow 0} \frac{\eta}{\epsilon} \int \left(p_{\text{data}} - \left[\mathcal{K}_\chi \right] - \epsilon \left[\mathcal{K}_\mathbf{d} \right] \right) \left(p_{\text{data}} - \left[\mathcal{K}_\chi \right] - \epsilon \left[\mathcal{K}_\mathbf{d} \right] \right)^{\dagger} - \left(p_{\text{data}} - \left[\mathcal{K}_\chi \right] \right) \left(p_{\text{data}} - \left[\mathcal{K}_\chi \right] \right)^{\dagger} \df{\xs} \df{\xr} \df{\omega} \\
& = & -\lim_{\epsilon \rightarrow 0} \frac{\eta}{\epsilon} \int \epsilon \left[ \left[\mathcal{K}_\mathbf{d} \right] \left(p_{\text{data}} - \left[\mathcal{K}_\chi \right]  \right)^{\dagger} + \left[\mathcal{K}_\mathbf{d} \right]^{\dagger} \left(p_{\text{data}} - \left[\mathcal{K}_\chi \right] \right) \right] - \epsilon^2 \left[\mathcal{K}_\mathbf{d} \right] \left[\mathcal{K}_\mathbf{d} \right]^{\dagger} \df{\xs} \df{\xr} \df{\omega} \\
& = & -2 \eta \int \real{\left[\mathcal{K}_\mathbf{d} \right] \left(p_{\text{data}} - \left[\mathcal{K}_\chi \right]  \right)^{\dagger}} \df{\xs} \df{\xr} \df{\omega} \\
& = & -2 \eta \int \real{\left[\mathcal{K}_\mathbf{d} \right](\xr, \xs, \omega)^{\dagger} r (\xr, \xs, \omega)} \df{\xs} \df{\xr} \df{\omega}
\end{eqnarray*}

For the discrete $\mathcal{K}$ operator the integrand will have the form

\[g_n(\vec{x}) = \eta \real{[\mathcal{K}^\star\mathbf{r}_{n-1}](\vec{x})} \]

where the $\star$ denote element wise multiplication per row and Re denotes that only the real part will be used. 
The adjoint operator $[\mathcal{K}^\star\mathbf{r}_{n-1}]$ can be seen as a backprojection operator that maps the residual between measured data and modelled data from the surface domain to its associated location in the scattering domain.

In our conjugate gradient scheme we make use of the Polak-Ribi$\grave{e}$re direction,

\[ \gamma_n = \frac{\int g_n(\vec{x})[g_n(\vec{x})-g_{n-1}(\vec{x})]d(\vec{x})}{\int g_{n-1}(\vec{x}) g_{n-1}(\vec{x})d(\vec{x})} \]

The optimal step size is found from the minimisation of cost functional equation, by setting the derivative equal to zero. Consequently the optimal step size becomes:

\[ \alpha_n = \frac {\real {\int \int \int r^{\star}_{n-1}(\mathbf{x_\text{r}},\mathbf{x_\text{s}},\omega)[\mathcal{K} \zeta_n](\mathbf{x_\text{r}},\mathbf{x_\text{s}},\omega)d\vec{x_s}d\vec{x_r}d\omega}}{\int \int \int \mid[\mathcal{K} \zeta_n](\mathbf{x_\text{r}},\mathbf{x_\text{s}},\omega) \mid^2 d\vec{x_s}d\vec{x_r}d\omega} \]

To initialise the conjugate gradient scheme, we assume, $\chi^0 (\vec{x}) = 0$, leading to $r_0(\mathbf{x_\text{r}},\mathbf{x_\text{s}},\omega) = p_data(\mathbf{x_\text{r}},\mathbf{x_\text{s}},\omega)$. Therefore, we find the gradient as :

\[g_1(\vec{x}) = \eta \real{[\mathcal{K}^\star p_\text{data}](\vec{x})} \]

and we get the first update parameter as :

\[ \alpha_1 = \frac {\real {\int \int \int p^{\star}_\text{data}(\mathbf{x_\text{r}},\mathbf{x_\text{s}},\omega)[\mathcal{K} g_1](\mathbf{x_\text{r}},\mathbf{x_\text{s}},\omega)d\vec{x_s}d\vec{x_r}d\omega}}{\int \int \int \mid[\mathcal{K} g_1](\mathbf{x_\text{r}},\mathbf{x_\text{s}},\omega) \mid^2 d\vec{x_s}d\vec{x_r}d\omega} \]


\subsection{Multiplicative Regularisation}

Since all seismic data contains some form of noise, the inversion process needs to be stabilised. Therefore, the CG scheme is extended in a way that it contains a multiplicative regularisation factor. 

The error functional $\mathcal{F}^tot$ becomes a product of the original error functional and a newly introduced regularisation factor $\mathcal{F}^reg$:

\[ \mathcal{F}^{tot}_n = \mathcal{F}^{data}_n \mathcal{F}^{reg}_n \]

The following weighting function is used:

\[ b_n^{2} = (\int_{\vec{x}} d\vec{x})^{-1} (\mid\nabla\chi^{n-1}\mid^2 + \delta_{n-1}^2)^{-1} \]

In the above equation, the $\delta_{n}^2$ is the steering factor and can be defined as:

\[ \delta_{n}^2 = (\int_{\vec{x}} d\vec{x})^{-1} \int_{\vec{x}} \mid\nabla\chi^{n-1}\mid^2 d\vec{x} \]

The new total cost functional then becomes a sum of two second order polynomials:

\[ \mathcal{F}^{tot}_n = (\mathcal{A}_2\alpha^{2}_n + \mathcal{A}_1\alpha_n +\mathcal{A}_0)(\mathcal{B}_2\alpha^{2}_n + \mathcal{B}_1\alpha_n + \mathcal{B}_0) \]

in which the constants are given by:

\[ \mathcal{A}_2 = \eta \int \int \int \mid \mathcal{K} \zeta_n \mid^2 d\vec{x_s}d\vec{x_r}d\omega \]

\[ \mathcal{A}_1 = -2 \eta \real{\int \int \int r^{\star}_{n-1} \mid \mathcal{K} \zeta_n \mid d\vec{x_s}d\vec{x_r}d\omega} , \hat{E} \]

\[ \mathcal{A}_0 = \eta \int \int \int \mid r_{n-1} \mid^2 d\vec{x_s}d\vec{x_r}d\omega =  \mathcal{F}^{data}_{n-1} , \hat{E} \]

\[ \mathcal{B}_2 = \mid \mid b_n \nabla \zeta_n \mid \mid^{2}_D , \hat{E} \]

\[ \mathcal{B}_1 = 2 < b_n \nabla\chi^{n-1}, b_n \nabla \zeta_n >_D, \hat{E} \]

\[ \mathcal{B}_0 = \mid \mid b_n \nabla \chi^{n-1} \mid \mid^{2}_D + \delta^{2}_{n-1} \mid \mid b_n \mid \mid^{2}_D \]

\subsection{Nonlinear field update based on the domain equation}

Initially, the subsurface properties are unknown  and the very first inversion is based on the assumption that wavefield propagation occurs in smooth non-scattering background medium only. Using an approximate smooth property model of the subsurface immediately tells us that the wavefield propagation in such an approximate medium cannot be exact and that for exact results we should make the wavefields consistent with the currently best estimate of the true medium.

Since the output of linear full waveform inversion is a property model, we do
not only obtain structural information but also a first order approximation of the subsurface properties at every grid point in the inversion domain. We can now use this property model to update the total field by solving the domain equation with in principle any suitable numerical method. Therefore, we iteratively build up the total fields as a sum of the background field and a number of basis functions:

\[ \phi_n (\mathbf{x_\text{r}},\mathbf{x_\text{s}},\omega) = \int_{\vec{x}\in D} \mathcal{G} (\mathbf{x_\text{r}},\mathbf{x_\text{s}},\omega)\partial \mathcal{W}_n d\vec{x}' \]

And the incremental contrast sources $\partial \mathcal{W}_n$ are given by:

\[ \partial \mathcal{W}_1 = \chi^{(1)} p^{0}_{tot} , \mathcal{W}_n = \chi^{(n)} p^{n-1}_{tot} - \chi^{(n-1)} p^{n-2}_{tot}, n > 1. \]

The weighting factors are then determined and $p_tot$ is updated according to the equation: 

\[ p^{(N)}_{tot} (\mathbf{x_\text{r}},\mathbf{x_\text{s}},\omega) =  p_0 (\mathbf{x_\text{r}},\mathbf{x_\text{s}},\omega) + \sum\limits_{N=1}^N \alpha^{(N)}_n (\mathbf{x_\text{s}},\omega) \phi_n (\mathbf{x_\text{r}},\mathbf{x_\text{s}},\omega) \]

\section{Computational Model}

\section{Code Algorithm}






\end{document}
