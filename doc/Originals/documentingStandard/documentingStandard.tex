\documentclass{article}
\title{Documenting standards}
\usepackage{graphicx}
\usepackage{fancyhdr}
\usepackage{hyperref}

% Adjust page margin
\usepackage[top=2.7cm, left=2cm, right=1.5cm, bottom=3.7cm]{geometry}

% Define ALTEN template (header and footer)
\pagestyle{fancy}
\renewcommand{\headrulewidth}{0pt}
\renewcommand{\footrulewidth}{0pt}
\lhead{\includegraphics[width=1.21cm]{./template/logoAlten.jpg}}
\chead{RESTRICTED}
\setlength{\headheight}{2cm}
\cfoot{\thepage}
\lfoot{\includegraphics[width=18cm]{./template/footerBar.jpg}}

\usepackage{verbatim}

\author{Alex Bijlsma}

\usepackage{listings}
\usepackage{xcolor}
\usepackage{hyperref}

\definecolor{dkgreen}{rgb}{0,0.6,0}
\definecolor{dred}{rgb}{0.545,0,0}
\definecolor{dblue}{rgb}{0,0,0.545}
\definecolor{lgrey}{rgb}{0.9,0.9,0.9}
\definecolor{gray}{rgb}{0.4,0.4,0.4}
\definecolor{darkblue}{rgb}{0.0,0.0,0.6}
\lstdefinelanguage{cpp}{
	backgroundcolor=\color{lgrey},  
	basicstyle=\footnotesize \ttfamily \color{black} \bfseries,   
	breakatwhitespace=false,       
	breaklines=true,               
	captionpos=b,                   
	commentstyle=\color{dkgreen},   
	deletekeywords={...},          
	escapeinside={\%*}{*)},                                    
	language=C++,                
	keywordstyle=\color{purple},  
	morekeywords={BRIEFDescriptorConfig,string,TiXmlNode,DetectorDescriptorConfigContainer,istringstream,cerr,exit}, 
	identifierstyle=\color{black},
	stringstyle=\color{blue},                     
	numbersep=5pt,                  
	numberstyle=\tiny\color{black}, 
	rulecolor=\color{black},        
	showspaces=false,               
	showstringspaces=false,        
	showtabs=false,                
	stepnumber=1,                   
	tabsize=5,                     
	title=\lstname,                 
}

\begin{document}
	
\maketitle

\section{Documenting program}
To document the written code, Doxygen will be used. Doxygen parses the sources and automatically generates documentation.

\section{Installation}
No prerequisite tools are needed to install Doxygen, Doxygen can be installed using the following command.\newline

\texttt{sudo apt-get install doxygen} \newline

To use Doxygen, a Doxyfile must be created/used. A Doxyfile is a configuration file for Doxygen.
To create a template Doxygen file, use the following command.\newline

\texttt{doxygen -g <config-file>}\newline

Where $<$config-file$>$ is the name of the Doxyfile.
In the folder of this .pdf, a pre-configured Doxyfile is given.
\section{Documenting standard}
\subsection{Comment blocks}
To create a comment block, the Javadoc style will be used. The starting line of a comment block starts with two *'s:

\begin{lstlisting}[language=cpp]
/**
 * .. text ..
 */
\end{lstlisting}

Comment blocks are used for i.e. functions and classes. These descriptions consist of a brief and detailed description. The detailed description ends at the first dot.
Keeping track of parameters, return value descriptions and reference is done using @.

\begin{lstlisting}[language=cpp]
/** Brief description of Foo(Bar). Detailed description of Foo(Bar). Type as much as you'd like
 *  here.
 *  @param Bar An integer argument
 *  @return A string
 *  @see Foo2();
 */
 string Foo(int Bar);
\end{lstlisting}

\subsection{Single line descriptions}
If you wish to document the members of a file, struct, union, class or enum, the documentation can be placed after the member instead of before. This is done when the documentation can be written down on the same line, by placing a < after the opening of a comment 'block': 
\begin{lstlisting}[language=cpp]
int var; /**< Detailed description of var */
\end{lstlisting}

\subsection{Special commands}
Special commands are used to document certain pieces of code with a certain syntax (i.e. structs/classes/enum/typedef/defines). A detailed list of the commands can be seen here:
\url{http://www.doxygen.nl/manual/commands.html}

\section{Generating HTML files}

To generate the HTML documentation, follow these steps:
\begin{itemize}
	\item Place the Doxyfile in the same directory as the CMakeLists.txt
	\item Run the following command in the command line:
	\texttt{doxygen <config-file>} where config-file is the name of the Doxyfile.
\end{itemize}
The generated documentation can be found in the 'html' folder in the same directory as the CMakeLists.txt

\end{document}