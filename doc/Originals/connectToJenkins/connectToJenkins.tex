\documentclass{article}
\title{How to connect to Jenkins from home}
\usepackage{graphicx}
\usepackage{fancyhdr}
\usepackage{hyperref}

% Adjust page margin
\usepackage[top=2.7cm, left=2cm, right=1.5cm, bottom=3.7cm]{geometry}

% Define ALTEN template (header and footer)
\pagestyle{fancy}
\renewcommand{\headrulewidth}{0pt}
\renewcommand{\footrulewidth}{0pt}
\lhead{\includegraphics[width=1.21cm]{./template/logoAlten.jpg}}
\chead{RESTRICTED}
\setlength{\headheight}{2cm}
\cfoot{\thepage}
\lfoot{\includegraphics[width=18cm]{./template/footerBar.jpg}}

\usepackage{verbatim}

\author{Alex Bijlsma}

\usepackage{listings}
\usepackage{xcolor}
\usepackage{hyperref}
\usepackage{url}
\usepackage{breakurl}

\newcommand\HREF[2]{\hyper@linkurl{#2}{#1}}

\begin{document}
\sloppy
	
\maketitle

\section{Setting up the VPN}
In order to set connect to Jenkins from home, you have to set up a VPN to the network of ALTEN. \\Execute the following steps:

\begin{enumerate}\setlength{\itemsep}{0pt}
	\item Perform the steps in this article:\\ \burl{https://servicedesk.alten.nl/support/solutions/articles/5000062308-vpn-verbinding\\-alten-nederland-ssl-vpn-}
	\item Open the following file as an administrator: \path{\C:\Windows\System32\drivers\etc\hosts}
	\\Note: it is recommended to open it in an editor such as Notepad++
	\item Add the following line at the bottom of the file:\newline \texttt{192.168.135.58 ci-full-waveform-inversion.alten.nl}
	\item Connect to the VPN on your Windows machine by logging in with your ALTEN account.
	\item Go to: \url{https://ci-full-waveform-inversion.alten.nl/}
	\item Log in with the following details:\\
	\texttt{Username: admin}\\
	\texttt{Password: admin}\\
\end{enumerate}

\end{document}